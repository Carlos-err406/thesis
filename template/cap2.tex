\chapter{Solución propuesta}\label{chap:2}

Incluir un resumen del contenido del capítulo en un párrafo.\\

\section{Título del epígrafe}

Texto... Es una buena práctica culminar (o iniciar) cada epígrafe con una referencia al que sigue (o precede) para dar fluidez a su lectura y no se vea como un \emph{copia y pega} sin ninguna relación.

\subsection{Título del subepígrafe}

Texto...

\subsection{Título del subepígrafe}

Texto...

\section{Título del epígrafe}

Texto...

\subsection{Título del subepígrafe}

Texto...

\subsection{Título del subepígrafe}

Texto...

\section{Conclusiones parciales}

% Cada conclusión tiene que estar sustentada en el cuerpo del capítulo.

Una vez terminado el capítulo se arriban a las siguientes conclusiones:

\begin{enumerate}
	\setlength\itemsep{0em}
	\item Unas conclusiones necesarias aquí son los aspectos tecnológicos que se tuvieron en cuenta para cumplir con los requisitos principales.
\end{enumerate}
\pagebreak

