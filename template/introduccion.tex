\chapter*{Introducción}
Describir el objeto de estudio con las definiciones fundamentales. No olvidar las referencias bibliográficas, aquí tienen un ejemplo \citep{pina2019incorporation}.\\

Las figuras deben referenciarse en el texto, así como lo muestra este ejemplo, ver Figura \ref{fig:figCUJAE}.

\begin{figure}[H] %la opción H indica al compilador LaTeX que posicione la figura lo más cerca posible de este lugar.
\centering
  \includegraphics[width=0.5\linewidth]{figuras/membrete-cujae-centrado.png}
  \caption{El título de la figura debe estar acorde con su contenido.}
  \label{fig:figCUJAE} %incluir el label permite referenciarla en cualquier parte del documento.
\end{figure}

Se deben utilizar siempre los mismos términos para referirse a los mismos conceptos y no olvidar de definir los términos que son claves en el campo de acción, o sea, la propuesta de la tesis.\\

Describir la Situación problemática. Problema. Objetivo general. Objetivos específicos. Tareas. Beneficios. Breve resumen del contenido de la tesis.