\begin{abstract}
The challenge of forming teams of appropriate software projects is of great importance for software development companies, given that the team plays an important role in the success or failure of a project. The present work consists of developing an application to manage the data corresponding to instances of said problem, the application is organized in three fundamental parts for the management of your data, the first where all the sets to be taken into account to form teams are taken into account, the second where the relationships established between said sets are found, and in the third part is where the user is allowed to select the data that they wish to have in the instance to be managed. It is important to note that the data corresponding to the relationships can be generated randomly following statistical distributions, besides the intancias can be generated from randomly choosing the sets that will belong to it. On the other hand it is possible to generate simple solutions to the problem in a random way and see its evaluation. The application also has a database where all the managed information is stored, which allows access to previously inserted instances to clone, modify or simply view them. This database can be exported or imported in Excel format. Python was used as programming language, Pycharm as development IDE, PyQt as visual development framework, JasperSoft for reporting and PostgreSQL as database manager.
\end{abstract}