\section{Capítulo 3 Validación de la solución}

Este capítulo presenta la validación de la solución de gestión de AD, mediante pruebas de lógica del cliente LDAP y flujos de usuario en la interfaz web. El objetivo es verificar que el sistema cumple con los requisitos funcionales y no funcionales.

\subsection{Metodología de pruebas}

Las pruebas en esta solución se llevaron a cabo utilizando las herramientas Vitest, Playwright y SonarQube. Mientras Vitest y Playwright, recomendadas por defecto en el framework SvelteKit, se enfocan en pruebas funcionales, SonarQube se integra en el proceso de despliegue para el análisis estático de código, permitiendo identificar problemas de calidad, seguridad y mantenibilidad del software. Este conjunto de herramientas proporciona un entorno robusto para garantizar que el sistema funcione correctamente bajo distintas condiciones, identificando y corrigiendo posibles errores.

Para un análisis más detallado de las pruebas realizadas, se puede consultar el \textit{Informe de Pruebas del Open Directory} \autocite{avangenio_srl_informe_nodate} (adjunto a la tesis), elaborado por Avangenio, el cliente de este producto. Este informe documenta los resultados obtenidos, los casos de prueba implementados y las conclusiones derivadas del proceso de validación.

La metodología de pruebas utilizada se basa principalmente en el enfoque de \textbf{pruebas de caja negra}, donde se evaluó el comportamiento del sistema sin necesidad de conocer su implementación interna. El objetivo fue garantizar que las entradas proporcionadas generaran las salidas esperadas, según los requisitos funcionales definidos.

Vitest fue la herramienta utilizada para implementar pruebas de \textbf{intrgación} enfocadas en el cliente LDAP, implementado con la librería ldapts. Estas pruebas validaron el comportamiento de las operaciones principales, como la búsqueda, adición, modificación y eliminación de entradas en el AD.

Por otro lado, Playwright se utilizó para realizar pruebas de \textbf{extremo a extremo} (\textit{E2E} por sus siglas en inglés) en los flujos de la interfaz de usuario. Estas pruebas evaluaron la interacción del usuario con la aplicación, simulando acciones como la creación, modificación y eliminación de usuarios. Se verificó que las notificaciones de éxito y error se mostraran correctamente, y que el sistema gestionara las sesiones de usuario de manera eficiente.

\textbf{Tipos de Pruebas}

Pruebas unitarias: se centraron en funciones individuales del cliente LDAP para asegurar que las operaciones básicas, como búsquedas y modificaciones, se comporten correctamente en distintos escenarios.

Pruebas E2E: a través de Playwright, se simularon interacciones completas de los usuarios con la interfaz, desde el inicio de sesión hasta la gestión de usuarios, garantizando que la aplicación responda adecuadamente a las solicitudes del usuario final.

\subsection{Pruebas}

El proceso de validación de la solución propuesta se llevó a cabo mediante un conjunto de pruebas que aseguran el correcto funcionamiento de las funcionalidades más relevantes del sistema. Dado el número considerable de casos de uso (19 en total), se ha priorizado la creación de tablas de casos de prueba para los casos de uso esenciales: creación y eliminación de usuarios, así como la autenticación. Estas tablas permiten organizar y documentar las diferentes condiciones y resultados esperados de cada operación, garantizando una cobertura adecuada de los escenarios más críticos para el sistema.

% 3 tables here

En la \autoref{fig:integration-tests-run-ok} y \autoref{fig:e2e-test-run-ok} se puede apreciar cómo las pruebas de integración realizadas con Vitest y las pruebas E2E ejecutadas con Playwright se llevaron a cabo con éxito. Durante el proceso de prueba, se detectaron varios errores en la implementación del manejo de errores en el cliente LDAP y en la validación de datos de entrada en los formularios. Estos errores, que inicialmente no eran evidentes en pruebas manuales, fueron identificados y corregidos gracias a la robustez de las pruebas implementadas.

La corrección de estos problemas permitió mejorar significativamente la confiabilidad del sistema. En particular, se mejoró la respuesta del cliente LDAP ante entradas no válidas y se reforzó la validación de datos en tiempo real, reduciendo así la posibilidad de fallos en la interacción con el AD. Estas mejoras contribuyen a una experiencia de usuario más estable y eficiente, cumpliendo con los requisitos de robustez y fiabilidad establecidos para la aplicación.


\textbf{Conclusiones parciales}\\
El proceso de validación demostró que la solución cumple satisfactoriamente con los requisitos funcionales y no funcionales establecidos. Las pruebas unitarias y de integración con Vitest validaron el correcto funcionamiento del cliente LDAP en operaciones críticas como creación, modificación y eliminación de usuarios. Las pruebas E2E con Playwright verificaron que los flujos de usuario en la interfaz web se comportan según lo esperado, incluyendo la autenticación y gestión de usuarios. Durante las pruebas se identificaron y corrigieron errores en el manejo de excepciones y validación de datos, mejorando la robustez del sistema. Los resultados obtenidos confirman que la aplicación es capaz de gestionar el AD a través de LDAP, cumpliendo con los criterios de calidad establecidos en términos de funcionalidad, usabilidad y estabilidad.
