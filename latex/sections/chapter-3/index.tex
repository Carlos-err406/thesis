\section{Capítulo 3 Validación de la solución}

Este capítulo presenta la validación de la solución de gestión de AD, mediante pruebas de lógica del cliente LDAP y flujos de usuario en la interfaz web. El objetivo es verificar que el sistema cumple con los requisitos funcionales y no funcionales.

\subsection{Metodología de pruebas}

Las pruebas en esta solución se llevaron a cabo utilizando las herramientas Vitest, Playwright y SonarQube. Mientras Vitest y Playwright, recomendadas por defecto en el framework SvelteKit, se enfocan en pruebas funcionales, SonarQube se integra en el proceso de despliegue para el análisis estático de código, permitiendo identificar problemas de calidad, seguridad y mantenibilidad del software. Este conjunto de herramientas proporciona un entorno robusto para garantizar que el sistema funcione correctamente bajo distintas condiciones, identificando y corrigiendo posibles errores.

Para un análisis más detallado de las pruebas realizadas, se puede consultar el \textit{Informe de Pruebas del Open Directory} \autocite{avangenio_srl_informe_nodate} (adjunto a la tesis), elaborado por Avangenio, el cliente de este producto. Este informe documenta los resultados obtenidos, los casos de prueba implementados y las conclusiones derivadas del proceso de validación.

La metodología de pruebas utilizada se basa principalmente en el enfoque de \textbf{pruebas de caja negra}, donde se evaluó el comportamiento del sistema sin necesidad de conocer su implementación interna. El objetivo fue garantizar que las entradas proporcionadas generaran las salidas esperadas, según los requisitos funcionales definidos.

Vitest fue la herramienta utilizada para implementar pruebas de \textbf{intrgación} enfocadas en el cliente LDAP, implementado con la librería ldapts. Estas pruebas validaron el comportamiento de las operaciones principales, como la búsqueda, adición, modificación y eliminación de entradas en el AD.

Por otro lado, Playwright se utilizó para realizar pruebas de \textbf{extremo a extremo} (\textit{E2E} por sus siglas en inglés) en los flujos de la interfaz de usuario. Estas pruebas evaluaron la interacción del usuario con la aplicación, simulando acciones como la creación, modificación y eliminación de usuarios. Se verificó que las notificaciones de éxito y error se mostraran correctamente, y que el sistema gestionara las sesiones de usuario de manera eficiente.

\textbf{Tipos de Pruebas}

Pruebas unitarias: se centraron en funciones individuales del cliente LDAP para asegurar que las operaciones básicas, como búsquedas y modificaciones, se comporten correctamente en distintos escenarios.

Pruebas E2E: a través de Playwright, se simularon interacciones completas de los usuarios con la interfaz, desde el inicio de sesión hasta la gestión de usuarios, garantizando que la aplicación responda adecuadamente a las solicitudes del usuario final.

\subsection{Pruebas}

El proceso de validación de la solución propuesta se llevó a cabo mediante un conjunto de pruebas que aseguran el correcto funcionamiento de las funcionalidades más relevantes del sistema. Dado el número considerable de casos de uso (19 en total), se ha priorizado la creación de tablas de casos de prueba para los casos de uso esenciales: creación y eliminación de usuarios, así como la autenticación. Estas tablas permiten organizar y documentar las diferentes condiciones y resultados esperados de cada operación, garantizando una cobertura adecuada de los escenarios más críticos para el sistema.


\begin{longtable}{|p{2cm}|p{2.5cm}|p{2.5cm}|p{2.5cm}|p{2.5cm}|p{2.5cm}|}
    \caption{Prueba del caso de uso eliminar usuario} \label{table:delete-user-test}                                                                                                                                                                                                                                                                                                                                          \\ % Caption and label at the start with a double backslash
    \hline
    \textbf{Caso de uso a probar}                                      & \multicolumn{5}{|l|}{Eliminar usuario}                                                                                                                                                                                                                                                                                                               \\ \hline
    \textbf{\seqsplit{Desarrollador}}                                  & \multicolumn{5}{|l|}{Carlos Daniel Vilaseca Illnait}                                                                                                                                                                                                                                                                                                 \\ \hline
    \textbf{Probador}                                                  & \multicolumn{5}{|l|}{Carlos Daniel Vilaseca Illnait}                                                                                                                                                                                                                                                                                                 \\ \hline
    \textbf{Fecha}                                                     & \multicolumn{5}{|l|}{5 de julio 2024}                                                                                                                                                                                                                                                                                                                \\ \hline
    \textbf{Version}                                                   & \multicolumn{5}{|l|}{1}                                                                                                                                                                                                                                                                                                                              \\ \hline
    \textbf{Objetivo de la Prueba}                                     & \multicolumn{5}{|p{13cm}|}{Comprobar que el usuario se elimina del directorio correctamente}                                                                                                                                                                                                                                                         \\ \hline
    \textbf{\seqsplit{Descripción} de la prueba}                       & \multicolumn{5}{|p{13cm}|}{El caso de uso de prueba inicia cuando se acepta la eliminación del usuario en la ventana de confirmación y culmina cuando el usuario queda eliminado del directorio.}                                                                                                                                                    \\ \hline
    \textbf{\seqsplit{Condiciones}}                                    & \multicolumn{5}{|p{13cm}|}{El usuario a eliminar existe en el directorio. El usuario a eliminar no es un objeto crítico del sistema.}                                                                                                                                                                                                                \\ \hline
    \multicolumn{4}{|l|}{\textbf{Combinaciones de valores de entrada}} & Resultado esperado                                                                                                                                                                                & Resultado obtenido                                                                                                                               \\ \hline
    \textbf{CP}                                                        & \textbf{Escenario}                                                                                                                                                                                & \textbf{Nombre de la variable de entrada} & \textbf{Valor} &                                          &                                          \\ \hline
    1                                                                  & El usuario no existe en el directorio                                                                                                                                                             & -                                         & -              & Error, el usuario no existe              & Error, el usuario no existe              \\ \hline
    2                                                                  & El usuario es un objeto crítico del sistema                                                                                                                                                       & -                                         & -              & Error, el usuario no puede ser eliminado & Error, el usuario no puede ser eliminado \\ \hline
    3                                                                  & El usuario existe y no es un objeto crítico del sistema                                                                                                                                           & -                                         & -              & Éxito, usuario eliminado                 & Éxito, usuario eliminado                 \\ \hline
    \multicolumn{6}{|l|}{\textbf{Observaciones}: }                                                                                                                                                                                                                                                                                                                                                                            \\ \hline
\end{longtable}

\begin{longtable}{|p{2cm}|p{2.5cm}|p{2.5cm}|p{2.5cm}|p{2.8cm}|p{2.8cm}|}
    \caption{Prueba del caso de uso crear usuario} \label{table:create-user-test}                                                                                                                                                                                                                                                                                                                                                                           \\
    \hline
    \textbf{Caso de uso a probar}                                      & \multicolumn{5}{|l|}{Crear usuario}                                                                                                                                                                                                                                                                                                                                                \\ \hline
    \textbf{\seqsplit{Desarrollador}}                                  & \multicolumn{5}{|l|}{Carlos Daniel Vilaseca Illnait}                                                                                                                                                                                                                                                                                                                               \\ \hline
    \textbf{Probador}                                                  & \multicolumn{5}{|l|}{Carlos Daniel Vilaseca Illnait}                                                                                                                                                                                                                                                                                                                               \\ \hline
    \textbf{Fecha}                                                     & \multicolumn{5}{|l|}{5 de julio 2024}                                                                                                                                                                                                                                                                                                                                              \\ \hline
    \textbf{Version}                                                   & \multicolumn{5}{|l|}{1}                                                                                                                                                                                                                                                                                                                                                            \\ \hline
    \textbf{Objetivo de la Prueba}                                     & \multicolumn{5}{|p{13cm}|}{Comprobar que el usuario se crea en el directorio correctamente}                                                                                                                                                                                                                                                                                        \\ \hline
    \textbf{\seqsplit{Descripción} de la prueba}                       & \multicolumn{5}{|p{13cm}|}{El caso de uso de prueba inicia cuando se accede al formulario de creación de usuario y culmina cuando el nuevo usuario queda creado en el directorio.}                                                                                                                                                                                                 \\ \hline
    \textbf{\seqsplit{Condiciones}}                                    & \multicolumn{5}{|p{13cm}|}{El usuario a crear no existe en el directorio. No se ha alcanzado el límite de usuarios.}                                                                                                                                                                                                                                                               \\ \hline
    \multicolumn{4}{|l|}{\textbf{Combinaciones de valores de entrada}} & Resultado esperado                                                                                                                                                                 & Resultado obtenido                                                                                                                                                                            \\ \hline
    \textbf{CP}                                                        & \textbf{Escenario}                                                                                                                                                                 & \textbf{Nombre de la variable de entrada} & \textbf{Valor}              &                                                          &                                                          \\ \hline
    1                                                                  & No se introduce el nombre                                                                                                                                                          & Nombre                                    & None                        & Error, el campo es requerido                             & Error, el campo es requerido                             \\ \hline
    2                                                                  & No se introduce la contraseña                                                                                                                                                      & Contraseña                                & None                        & Error. el campo es requerido                             & Error. el campo es requerido                             \\ \hline
    3                                                                  & No se introduce el nombre de usuario                                                                                                                                               & Nombre de usuario                         & None                        & Error. el campo es requerido                             & Error. el campo es requerido                             \\ \hline
    4                                                                  & No se introduce la confirmación de contraseña                                                                                                                                      & Confirmación de contraseña                & None                        & Error. Las contraseñas deben coincidir                   & Error. Las contraseñas deben coincidir                   \\ \hline
    5                                                                  & No se introduce el correo                                                                                                                                                          & Correo                                    & None                        & Error. el campo es requerido                             & Error. el campo es requerido                             \\ \hline
    6                                                                  & La contraseña no tiene dígitos                                                                                                                                                     & Contraseña                                & \$inNumero\$                & La contraseña debe incluir al menos un número            & La contraseña debe incluir al menos un número            \\ \hline
    7                                                                  & La contraseña no llega a los 8 caracteres                                                                                                                                          & Contraseña                                & C0rt@                       & La contraseña debe tener al menos 8 caracteres           & La contraseña debe tener al menos 8 caracteres           \\ \hline
    8                                                                  & La contraseña no tiene caracteres especiales                                                                                                                                       & Contraseña                                & \seqsplit{SinC4r4cteresEsp} & La contraseña debe incluir al menos un caracter especial & La contraseña debe incluir al menos un caracter especial \\ \hline
    9                                                                  & La contraseña no tiene mayúsculas                                                                                                                                                  & Contraseña                                & \seqsplit{s1n\_mayusculas}  & La contraseña debe incluir al menos una mayúscula        & La contraseña debe incluir al menos una mayúscula        \\ \hline
    10                                                                 & La contraseña no tiene minúsculas                                                                                                                                                  & Contraseña                                & \seqsplit{S1N\_MINUSCULAS}  & La contraseña debe incluir al menos una minúscula        & La contraseña debe incluir al menos una minúscula        \\ \hline
    11                                                                 & Usuario ya existe                                                                                                                                                                  & Nombre de usuario                         & \seqsplit{Administrator}    & Error: El usuario ya existe                              & Error: El usuario ya existe                              \\ \hline
    12                                                                 & El correo no es válido                                                                                                                                                             & Correo                                    & \seqsplit{user@domain}      & Correo inválido                                          & Correo inválido                                          \\ \cline{4-4}
                                                                       &                                                                                                                                                                                    &                                           & \seqsplit{user@domain..com} &                                                          &                                                          \\ \cline{4-4}
                                                                       &                                                                                                                                                                                    &                                           & \seqsplit{user@.domain.com} &                                                          &                                                          \\ \cline{4-4}
                                                                       &                                                                                                                                                                                    &                                           & \seqsplit{user@domain@.com} &                                                          &                                                          \\ \cline{4-4}
                                                                       &                                                                                                                                                                                    &                                           & \seqsplit{user@domain!com}  &                                                          &                                                          \\ \hline
    13                                                                 & Las contraseñas no coinciden                                                                                                                                                       & Contraseña                                & Def password(1)             & Las contraseñas no coinciden                             & Las contraseñas no coinciden                             \\ \cline{3-4}
                                                                       &                                                                                                                                                                                    & Conf. de contraseña                       & Def password()              &                                                          &                                                          \\ \hline
    14                                                                 & Datos requeridos correctos                                                                                                                                                         & Nombre                                    & Carlos Daniel               & Usuario creado \seqsplit{exitosamente}                   & Usuario creado \seqsplit{exitosamente}                   \\ \cline{3-4}
                                                                       &                                                                                                                                                                                    & Nombre de usuario                         & charlie.01                  &                                                          &                                                          \\ \cline{3-4}
                                                                       &                                                                                                                                                                                    & Correo                                    & \seqsplit{carlosd@comp.cu}  &                                                          &                                                          \\ \cline{3-4}
                                                                       &                                                                                                                                                                                    & Contraseña                                & dsc!QK33                    &                                                          &                                                          \\ \cline{3-4}
                                                                       &                                                                                                                                                                                    & Conf. de contraseña                       & dsc!QK33                    &                                                          &                                                          \\ \hline


    \multicolumn{6}{|l|}{\textbf{Observaciones}: }                                                                                                                                                                                                                                                                                                                                                                                                          \\ \hline
\end{longtable}


\begin{longtable}{|p{2cm}|p{2.5cm}|p{2.5cm}|p{2.5cm}|p{2.5cm}|p{2.7cm}|}
    \caption{Prueba del caso de uso autenticarse} \label{table:authentication-test}                                                                                                                                                                                                                                                                                                                                                                                                              \\
    \hline
    \textbf{Caso de uso a probar}                                      & \multicolumn{5}{|l|}{Eliminar usuario}                                                                                                                                                                                                                                                                                                                                                                                  \\ \hline
    \textbf{\seqsplit{Desarrollador}}                                  & \multicolumn{5}{|l|}{Carlos Daniel Vilaseca Illnait}                                                                                                                                                                                                                                                                                                                                                                    \\ \hline
    \textbf{Probador}                                                  & \multicolumn{5}{|l|}{Carlos Daniel Vilaseca Illnait}                                                                                                                                                                                                                                                                                                                                                                    \\ \hline
    \textbf{Fecha}                                                     & \multicolumn{5}{|l|}{5 de julio 2024}                                                                                                                                                                                                                                                                                                                                                                                   \\ \hline
    \textbf{Version}                                                   & \multicolumn{5}{|l|}{1}                                                                                                                                                                                                                                                                                                                                                                                                 \\ \hline
    \textbf{Objetivo de la Prueba}                                     & \multicolumn{5}{|p{13cm}|}{Comprobar que los usuarios pueden autenticarse en el sistema}                                                                                                                                                                                                                                                                                                                                \\ \hline
    \textbf{\seqsplit{Descripción} de la prueba}                       & \multicolumn{5}{|p{13cm}|}{El caso de uso de prueba inicia cuando se accede a la aplicación y se muestra el formulario de autenticación y culmina cuando el usuario queda autenticado en el sistema y se encuentra en la página de su perfil.}                                                                                                                                                                          \\ \hline
    \textbf{\seqsplit{Condiciones}}                                    & \multicolumn{5}{|p{13cm}|}{Los credenciales y el captcha deben ser correctos.}                                                                                                                                                                                                                                                                                                                                          \\ \hline
    \multicolumn{4}{|l|}{\textbf{Combinaciones de valores de entrada}} & Resultado esperado                                                                                                                                                                                                                             & Resultado obtenido                                                                                                                                                     \\ \hline
    \textbf{CP}                                                        & \textbf{Escenario}                                                                                                                                                                                                                             & \textbf{Nombre de la variable de entrada} & \textbf{Valor}                 &                                             &                                             \\ \hline
    1                                                                  & Se resuelve el captcha \seqsplit{incorrectamente}                                                                                                                                                                                              & captcha                                   & ***                            & Error, Captcha incorrecto                   & Error, Captcha incorrecto                   \\ \hline
    2                                                                  & Las credenciales son inválidos                                                                                                                                                                                                                 & correo                                    & \seqsplit{some.email@mail.com} & Error, credenciales inválidas               & Error, credenciales inválidas               \\ \cline{3-4}
                                                                       &                                                                                                                                                                                                                                                & contraseña                                & 123                            &                                             &                                             \\ \hline
    3                                                                  & Las credenciales son válidas, pero la contraseña expiró                                                                                                                                                                                        & correo                                    & \seqsplit{some.email@mail.com} & Error, contraseña expirada                  & Error, contraseña expirada                  \\ \cline{3-4}
                                                                       &                                                                                                                                                                                                                                                & contraseña                                & \seqsplit{cOrrEctP@assw0rd}    &                                             &                                             \\ \hline
    4                                                                  & Las credenciales son válidas, pero la cuenta está deshabilitada                                                                                                                                                                                & correo                                    & \seqsplit{some.email@mail.com} & Error, su cuenta se encuentra deshabilitada & Error, su cuenta se encuentra deshabilitada \\ \cline{3-4}
                                                                       &                                                                                                                                                                                                                                                & contraseña                                & \seqsplit{cOrrEctP@assw0rd}    &                                             &                                             \\ \hline
    5                                                                  & Las credenciales son válidas                                                                                                                                                                                                                   & correo                                    & \seqsplit{some.email@mail.com} & Éxito, sesion iniciada                      & Éxito, sesion iniciada                      \\ \cline{3-4}
                                                                       &                                                                                                                                                                                                                                                & contraseña                                & \seqsplit{cOrrEctP@assw0rd}    &                                             &                                             \\ \hline

    \multicolumn{6}{|l|}{\textbf{Observaciones}: }                                                                                                                                                                                                                                                                                                                                                                                                                                               \\ \hline
\end{longtable}


Durante el proceso de prueba, se detectaron varios errores en la implementación del manejo de errores en el cliente LDAP y en la validación de datos de entrada en los formularios. Estos errores, que inicialmente no eran evidentes en pruebas manuales, fueron identificados y corregidos gracias a la robustez de las pruebas implementadas.

La corrección de estos problemas permitió mejorar significativamente la confiabilidad del sistema. En particular, se mejoró el manejo de las respuestas del cliente LDAP ante entradas no válidas y se reforzó la validación de datos en tiempo real, reduciendo así la posibilidad de fallos en la interacción con el AD. Estas mejoras contribuyen a una experiencia de usuario más estable y eficiente, cumpliendo con los requisitos de robustez y fiabilidad establecidos para la aplicación.


\textbf{Conclusiones parciales}\\
El proceso de validación demostró que la solución cumple satisfactoriamente con los requisitos funcionales y no funcionales establecidos. Las pruebas unitarias y de integración con Vitest validaron el correcto funcionamiento del cliente LDAP en operaciones críticas como creación, modificación y eliminación de usuarios. Las pruebas E2E con Playwright verificaron que los flujos de usuario en la interfaz web se comportan según lo esperado, incluyendo la autenticación y gestión de usuarios. Durante las pruebas se identificaron y corrigieron errores en el manejo de excepciones y validación de datos, mejorando la robustez del sistema. Los resultados obtenidos confirman que la aplicación es capaz de gestionar el AD a través de LDAP, cumpliendo con los criterios de calidad establecidos.
