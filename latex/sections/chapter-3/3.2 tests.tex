\subsection{Pruebas}

El proceso de validación de la solución propuesta se llevó a cabo mediante un conjunto de pruebas que aseguran el correcto funcionamiento de las funcionalidades más relevantes del sistema. Dado el número considerable de casos de uso (19 en total), se ha priorizado la creación de tablas de casos de prueba para los casos de uso esenciales: creación y eliminación de usuarios, así como la autenticación. Estas tablas permiten organizar y documentar las diferentes condiciones y resultados esperados de cada operación, garantizando una cobertura adecuada de los escenarios más críticos para el sistema.

% 3 tables here

En la \autoref{fig:integration-tests-run-ok} y \autoref{fig:e2e-test-run-ok} se puede apreciar cómo las pruebas de integración realizadas con Vitest y las pruebas E2E ejecutadas con Playwright se llevaron a cabo con éxito. Durante el proceso de prueba, se detectaron varios errores en la implementación del manejo de errores en el cliente LDAP y en la validación de datos de entrada en los formularios. Estos errores, que inicialmente no eran evidentes en pruebas manuales, fueron identificados y corregidos gracias a la robustez de las pruebas implementadas.

La corrección de estos problemas permitió mejorar significativamente la confiabilidad del sistema. En particular, se mejoró la respuesta del cliente LDAP ante entradas no válidas y se reforzó la validación de datos en tiempo real, reduciendo así la posibilidad de fallos en la interacción con el AD. Estas mejoras contribuyen a una experiencia de usuario más estable y eficiente, cumpliendo con los requisitos de robustez y fiabilidad establecidos para la aplicación.
