\subsubsection{Frameworks para el desarrollo web}

En el ámbito del desarrollo web, los frameworks proporcionan estructuras y herramientas que simplifican y aceleran la creación de aplicaciones. Estos frameworks ofrecen una base sólida para la implementación de funcionalidades complejas, facilitando la interacción entre el diseño front-end y la lógica de negocio back-end. Esta sección explora diversos frameworks destacados en el panorama actual, analizando sus características, ventajas y aplicaciones específicas en el desarrollo web moderno.

La \autoref{table:web-frameworks-comparison} presenta una comparación detallada de tres frameworks populares en el desarrollo web: SvelteKit, Next.js y Nuxt.js. Esta comparación se enfoca en varias características clave que influyen en la elección de un framework para proyectos web, tales como eficiencia, flexibilidad, escalabilidad, curva de aprendizaje y comunidad. Al analizar estos aspectos, se puede obtener una visión más clara de las fortalezas y debilidades de cada framework, ayudando a los desarrolladores a tomar decisiones informadas al seleccionar la herramienta más adecuada para sus necesidades específicas \autocite{riva_real-world_2022,lazuardy_2022_modern,kok_hands-nuxt_2020,halliday_vue_2018,wernersson_choosing_2023,kroon_celander_comparative_2024,c_ragkhitwetsagul_jscefr_2024,vepsalainen_state_2023}.

\begin{longtable}{|p{3cm}|p{3cm}|p{4cm}|p{5cm}|}
    \caption{Comparación entre frameworks de desarrollo web}
    \label{table:web-frameworks-comparison}                                                                                                                                                                                                                                                                                    \\
    \hline
    \textbf{Característica}       & \textbf{Sveltekit}                                         & \textbf{Next.js}                                                                      & \textbf{Nuxt.js}                                                                                                                      \\
    \hline
    \endfirsthead
    \textbf{Eficiencia}           & Alto rendimiento con compilación previa y sitios estáticos & Buena eficiencia con renderizado del lado del servidor y en el cliente                & Eficiencia en el desarrollo de aplicaciones web, con facilidades para la creación de aplicaciones universal y estáticamente generadas \\
    \hline
    \textbf{Flexibilidad}         & Gran flexibilidad en personalización y configuración       & Flexible para la creación de diferentes tipos de aplicaciones web                     & Flexible y adaptable a diferentes tipos de proyectos, con un enfoque en la simplicidad y facilidad de uso                             \\
    \hline
    \textbf{Escalabilidad}        & Altamente escalable para proyectos de diferentes tamaños   & Buena capacidad de escalar y manejar proyectos de gran envergadura                    & Puede escalar adecuadamente para manejar proyectos de diversos tamaños y complejidades                                                \\
    \hline
    \textbf{Curva de aprendizaje} & Muy baja, con sintaxis simple y familiar                   & Moderada, requiere familiarizarse con sus conceptos y funcionalidades                 & Moderada, especialmente para aquellos que están familiarizados con Vue.js                                                             \\
    \hline
    \textbf{Comunidad}            & En crecimiento, con soporte activo y recursos disponibles  & Amplia comunidad de desarrolladores, con gran cantidad de recursos y soporte en línea & Comunidad activa y en crecimiento, con soporte y recursos disponibles                                                                 \\
    \hline
\end{longtable}

Es importante destacar que realizar una comparación exhaustiva de todos los frameworks disponibles es complicado por varias razones:

\begin{itemize}
    \item Abundancia de opciones: Actualmente, existen cientos de frameworks de desarrollo web, cada uno con características y ventajas únicas.
    \item Complejidad inherente: Los frameworks son complejos y ofrecen una amplia gama de características, lo que dificulta una comparación exhaustiva y detallada.
    \item Variabilidad en los requisitos: Las aplicaciones web tienen requisitos diversos, lo que significa que un framework ideal para una aplicación puede no serlo para otra.
\end{itemize}

Por lo tanto, es difícil determinar que un framework sea superior a otro de manera generalizada. La elección del mejor framework para una aplicación específica depende en gran medida de los requisitos particulares de cada aplicación y de las preferencias del equipo de desarrollo. En última instancia, la selección de un framework adecuado se basa en su capacidad para resolver los problemas específicos de desarrollo que se presentan en el contexto de la aplicación deseada.
