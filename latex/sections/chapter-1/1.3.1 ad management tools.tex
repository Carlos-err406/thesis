\subsubsection{Herramientas de gestión de AD}

Para gestionar efectivamente un Directorio Activo (AD), es crucial contar con herramientas que simplifiquen las tareas administrativas y ofrezcan opciones flexibles de personalización y configuración. La capacidad de personalización se refiere a la flexibilidad que ofrece la herramienta para ajustar su interfaz y funcionalidades según las necesidades específicas del usuario u organización \autocite{van_der_hoek_configurable_1999}. Por otro lado, la facilidad de uso está relacionada con la simplicidad con la que una herramienta puede ser operada y configurada, asegurando una experiencia de administración eficiente y sin complicaciones \autocite{sheppard_re-examining_2019}.

En la \autoref{table:ad-tech-comparison} se presentan algunas herramientas para la gestión de AD, resaltando sus características principales en términos de personalización y facilidad de uso \autocite{graber_stgrabersamba4-manager_2024,jerez_vicentgjad-webmanager_2024,han_remote_2024,karzynski_webmin_2014}.

\begin{longtable}{|l|p{5cm}|p{5cm}|}
    \caption{Comparación de tecnologías existentes en cuanto a capacidad de personalización y facilidad de uso
    }
    \label{table:ad-tech-comparison}                                                                                                                                                                                                                                                                       \\
    \hline
    \textbf{Herrameinta} & \textbf{Capacidad de personalización (interfaz y apariencia)}                                                                                   & \textbf{Facilidad de uso}                                                                                                     \\
    \hline
    \endfirsthead
    \hline
    RSAT                 & Limitada, la personalización se limita a ajustes mínimos dentro del entorno de Windows                                                          & Alta, ya que es familiar para administradores de Windows, pero requiere conocimientos previos de AD                           \\
    \hline
    Webmin               & Moderada, permite cierta personalización a través de temas y ajustes de interfaz, pero con limitaciones en la profundidad de las modificaciones & Moderada, la interfaz es intuitiva, pero la configuración de módulos puede ser compleja para usuarios sin experiencia técnica \\
    \hline
    samba4-manager       & Limitada, diseñada específicamente para la gestión de Samba4, con opciones de personalización limitadas                                         & Baja, debido a la complejidad de la configuración y el mantenimiento en entornos no homogéneos                                \\
    \hline
    ADWebmanager         & Limitada, diseñada para funciones comunes de AD, con mínimas opciones de personalización de interfaz                                            & Alta, diseñada para simplificar tareas comunes de gestión de AD, con una curva de aprendizaje reducida                        \\
    \hline
\end{longtable}
