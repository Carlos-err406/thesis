\section{Capítulo 1 Fundamentación teórica}
% \addcontentsline{toc}{section}{Capítulo 1 Fundamentación teórica}

El propósito de este capítulo es proporcionar una base sólida sobre los conceptos y tecnologías fundamentales para la gestión de usuarios y AD, que sustentan la propuesta de solución planteada. Se explorará la importancia de una gestión eficaz de usuarios en entornos digitales, abordando los principios clave de seguridad y autenticación que son esenciales para proteger la integridad y confidencialidad de los datos. Asimismo, se detallarán los conceptos de AD y LDAP, explicando su funcionamiento y relevancia en la administración de identidades y accesos. Además, se realizará un análisis exhaustivo de las herramientas existentes para la gestión de AD, evaluando sus ventajas y limitaciones, con el objetivo de establecer un marco teórico robusto que guíe el desarrollo de una solución personalizable y fácil de desplegar.

\input{sections/chapter-1/1.1}
\subsection{Protocolo Ligero de Acceso a Directorios (LDAP)}

LDAP es un estándar de protocolo utilizado para acceder y gestionar información almacenada en directorios de manera eficiente y segura. En el contexto de la administración de sistemas informáticos, LDAP desempeña un papel fundamental al facilitar la búsqueda, autenticación y gestión de usuarios, dispositivos y otros recursos \autocite{sermersheim_lightweight_2006,carter_ldap_2003}.

Este epígrafe explora los principios fundamentales de LDAP, incluyendo su arquitectura, funcionamiento y principales características. Se examinará cómo LDAP permite la estructuración jerárquica de la información mediante la utilización de entradas y atributos, lo cual facilita la organización y el acceso a datos.

\textbf{Arquitectura de LDAP}

LDAP se basa en una arquitectura cliente-servidor, donde el cliente LDAP envía solicitudes al servidor LDAP para realizar diversas operaciones, como búsquedas, actualizaciones y autenticaciones de información almacenada en el directorio. Esta arquitectura facilita la gestión centralizada y eficiente de los datos \autocite{harrison_lightweight_2006,sermersheim_lightweight_2006,carter_ldap_2003}.

\begin{itemize}
    \item \textbf{Cliente LDAP}: Es el software que realiza peticiones de búsqueda, modificación o consulta de información almacenada en el servidor LDAP.
    \item \textbf{Servidor LDAP}: Es el software que almacena la base de datos de directorio y responde a las peticiones de los clientes LDAP. El servidor LDAP gestiona y organiza la información en forma de entradas almacenadas en un árbol de directorio.
    \item \textbf{Protocolo de Comunicación}: LDAP define cómo se comunican el cliente y el servidor a través de un protocolo eficiente y ligero, diseñado principalmente para la lectura, búsqueda y modificación de información en directorios.
    \item \textbf{Eficiencia}: permite búsquedas rápidas y eficientes de información en grandes volúmenes de datos.
    \item \textbf{Seguridad}: soporta protocolos de seguridad como TLS (Transport Layer Security) para proteger la integridad y confidencialidad de los datos transmitidos.
    \item \textbf{Escalabilidad}: capacidad para manejar grandes cantidades de datos y usuarios dentro de un directorio, adaptándose a las necesidades de crecimiento de una organización.
    \item \textbf{Interoperabilidad}: estándar abierto compatible con una amplia gama de plataformas y sistemas de directorio.
\end{itemize}


\textbf{Aplicaciones de LDAP}

LDAP se utiliza ampliamente en la autenticación de usuarios, control de acceso y gestión de identidades en sistemas operativos, aplicaciones web y servicios de correo electrónico. Su flexibilidad y robustez lo convierten en una herramienta fundamental para la integración y administración de infraestructuras de TI empresariales \autocite{sermersheim_lightweight_2006,redhat_what_2022,carter_ldap_2003}.



\subsection{Tecnologías y herramientas existentes}

En el ámbito de la gestión de AD, existen diversas tecnologías y herramientas diseñadas para facilitar esta tarea esencial en la administración de sistemas informáticos. Estas herramientas no solo permiten una gestión más eficiente de los recursos y usuarios dentro de una organización, sino que también contribuyen a mejorar la seguridad y el control de acceso.

Este epígrafe se centrará en revisar algunas de las herramientas existentes para la gestión de AD, detallando sus principales funcionalidades y características. Asimismo, se analizarán las ventajas y limitaciones de estas soluciones.


\subsubsection{Herramientas de gestión de AD}

Para gestionar efectivamente un Directorio Activo (AD), es crucial contar con herramientas que simplifiquen las tareas administrativas y ofrezcan opciones flexibles de personalización y configuración. La capacidad de personalización se refiere a la flexibilidad que ofrece la herramienta para ajustar su interfaz y funcionalidades según las necesidades específicas del usuario u organización \autocite{van_der_hoek_configurable_1999}. Por otro lado, la facilidad de uso está relacionada con la simplicidad con la que una herramienta puede ser operada y configurada, asegurando una experiencia de administración eficiente y sin complicaciones \autocite{sheppard_re-examining_2019}.

En la \autoref{table:ad-tech-comparison} se presentan algunas herramientas para la gestión de AD, resaltando sus características principales en términos de personalización y facilidad de uso \autocite{graber_stgrabersamba4-manager_2024,jerez_vicentgjad-webmanager_2024,han_remote_2024,karzynski_webmin_2014}.

\begin{longtable}{|l|p{5cm}|p{5cm}|}
    \caption{Comparación de tecnologías existentes en cuanto a capacidad de personalización y facilidad de uso
    }
    \label{table:ad-tech-comparison}                                                                                                                                                                                                                                                                       \\
    \hline
    \textbf{Herrameinta} & \textbf{Capacidad de personalización (interfaz y apariencia)}                                                                                   & \textbf{Facilidad de uso}                                                                                                     \\
    \hline
    \endfirsthead
    \hline
    RSAT                 & Limitada, la personalización se limita a ajustes mínimos dentro del entorno de Windows                                                          & Alta, ya que es familiar para administradores de Windows, pero requiere conocimientos previos de AD                           \\
    \hline
    Webmin               & Moderada, permite cierta personalización a través de temas y ajustes de interfaz, pero con limitaciones en la profundidad de las modificaciones & Moderada, la interfaz es intuitiva, pero la configuración de módulos puede ser compleja para usuarios sin experiencia técnica \\
    \hline
    samba4-manager       & Limitada, diseñada específicamente para la gestión de Samba4, con opciones de personalización limitadas                                         & Baja, debido a la complejidad de la configuración y el mantenimiento en entornos no homogéneos                                \\
    \hline
    ADWebmanager         & Limitada, diseñada para funciones comunes de AD, con mínimas opciones de personalización de interfaz                                            & Alta, diseñada para simplificar tareas comunes de gestión de AD, con una curva de aprendizaje reducida                        \\
    \hline
\end{longtable}

\subsection{Tecnologías y herramientas existentes}

En el ámbito de la gestión de AD, existen diversas tecnologías y herramientas diseñadas para facilitar esta tarea esencial en la administración de sistemas informáticos. Estas herramientas no solo permiten una gestión más eficiente de los recursos y usuarios dentro de una organización, sino que también contribuyen a mejorar la seguridad y el control de acceso.

Este epígrafe se centrará en revisar algunas de las herramientas existentes para la gestión de AD, detallando sus principales funcionalidades y características. Asimismo, se analizarán las ventajas y limitaciones de estas soluciones.


\subsubsection{Herramientas de gestión de AD}

Para gestionar efectivamente un Directorio Activo (AD), es crucial contar con herramientas que simplifiquen las tareas administrativas y ofrezcan opciones flexibles de personalización y configuración. La capacidad de personalización se refiere a la flexibilidad que ofrece la herramienta para ajustar su interfaz y funcionalidades según las necesidades específicas del usuario u organización \autocite{van_der_hoek_configurable_1999}. Por otro lado, la facilidad de uso está relacionada con la simplicidad con la que una herramienta puede ser operada y configurada, asegurando una experiencia de administración eficiente y sin complicaciones \autocite{sheppard_re-examining_2019}.

En la \autoref{table:ad-tech-comparison} se presentan algunas herramientas para la gestión de AD, resaltando sus características principales en términos de personalización y facilidad de uso \autocite{graber_stgrabersamba4-manager_2024,jerez_vicentgjad-webmanager_2024,han_remote_2024,karzynski_webmin_2014}.

\begin{longtable}{|l|p{5cm}|p{5cm}|}
    \caption{Comparación de tecnologías existentes en cuanto a capacidad de personalización y facilidad de uso
    }
    \label{table:ad-tech-comparison}                                                                                                                                                                                                                                                                       \\
    \hline
    \textbf{Herrameinta} & \textbf{Capacidad de personalización (interfaz y apariencia)}                                                                                   & \textbf{Facilidad de uso}                                                                                                     \\
    \hline
    \endfirsthead
    \hline
    RSAT                 & Limitada, la personalización se limita a ajustes mínimos dentro del entorno de Windows                                                          & Alta, ya que es familiar para administradores de Windows, pero requiere conocimientos previos de AD                           \\
    \hline
    Webmin               & Moderada, permite cierta personalización a través de temas y ajustes de interfaz, pero con limitaciones en la profundidad de las modificaciones & Moderada, la interfaz es intuitiva, pero la configuración de módulos puede ser compleja para usuarios sin experiencia técnica \\
    \hline
    samba4-manager       & Limitada, diseñada específicamente para la gestión de Samba4, con opciones de personalización limitadas                                         & Baja, debido a la complejidad de la configuración y el mantenimiento en entornos no homogéneos                                \\
    \hline
    ADWebmanager         & Limitada, diseñada para funciones comunes de AD, con mínimas opciones de personalización de interfaz                                            & Alta, diseñada para simplificar tareas comunes de gestión de AD, con una curva de aprendizaje reducida                        \\
    \hline
\end{longtable}

\subsection{Tecnologías y herramientas existentes}

En el ámbito de la gestión de AD, existen diversas tecnologías y herramientas diseñadas para facilitar esta tarea esencial en la administración de sistemas informáticos. Estas herramientas no solo permiten una gestión más eficiente de los recursos y usuarios dentro de una organización, sino que también contribuyen a mejorar la seguridad y el control de acceso.

Este epígrafe se centrará en revisar algunas de las herramientas existentes para la gestión de AD, detallando sus principales funcionalidades y características. Asimismo, se analizarán las ventajas y limitaciones de estas soluciones.


\input{sections/chapter-1/1.3.1}
\input{sections/chapter-1/1.3.2}
\input{sections/chapter-1/1.3.3}
\subsection{Tecnologías y herramientas existentes}

En el ámbito de la gestión de AD, existen diversas tecnologías y herramientas diseñadas para facilitar esta tarea esencial en la administración de sistemas informáticos. Estas herramientas no solo permiten una gestión más eficiente de los recursos y usuarios dentro de una organización, sino que también contribuyen a mejorar la seguridad y el control de acceso.

Este epígrafe se centrará en revisar algunas de las herramientas existentes para la gestión de AD, detallando sus principales funcionalidades y características. Asimismo, se analizarán las ventajas y limitaciones de estas soluciones.


\input{sections/chapter-1/1.3.1}
\input{sections/chapter-1/1.3.2}
\input{sections/chapter-1/1.3.3}
\subsection{Tecnologías y herramientas existentes}

En el ámbito de la gestión de AD, existen diversas tecnologías y herramientas diseñadas para facilitar esta tarea esencial en la administración de sistemas informáticos. Estas herramientas no solo permiten una gestión más eficiente de los recursos y usuarios dentro de una organización, sino que también contribuyen a mejorar la seguridad y el control de acceso.

Este epígrafe se centrará en revisar algunas de las herramientas existentes para la gestión de AD, detallando sus principales funcionalidades y características. Asimismo, se analizarán las ventajas y limitaciones de estas soluciones.


\subsubsection{Herramientas de gestión de AD}

Para gestionar efectivamente un Directorio Activo (AD), es crucial contar con herramientas que simplifiquen las tareas administrativas y ofrezcan opciones flexibles de personalización y configuración. La capacidad de personalización se refiere a la flexibilidad que ofrece la herramienta para ajustar su interfaz y funcionalidades según las necesidades específicas del usuario u organización \autocite{van_der_hoek_configurable_1999}. Por otro lado, la facilidad de uso está relacionada con la simplicidad con la que una herramienta puede ser operada y configurada, asegurando una experiencia de administración eficiente y sin complicaciones \autocite{sheppard_re-examining_2019}.

En la \autoref{table:ad-tech-comparison} se presentan algunas herramientas para la gestión de AD, resaltando sus características principales en términos de personalización y facilidad de uso \autocite{graber_stgrabersamba4-manager_2024,jerez_vicentgjad-webmanager_2024,han_remote_2024,karzynski_webmin_2014}.

\begin{longtable}{|l|p{5cm}|p{5cm}|}
    \caption{Comparación de tecnologías existentes en cuanto a capacidad de personalización y facilidad de uso
    }
    \label{table:ad-tech-comparison}                                                                                                                                                                                                                                                                       \\
    \hline
    \textbf{Herrameinta} & \textbf{Capacidad de personalización (interfaz y apariencia)}                                                                                   & \textbf{Facilidad de uso}                                                                                                     \\
    \hline
    \endfirsthead
    \hline
    RSAT                 & Limitada, la personalización se limita a ajustes mínimos dentro del entorno de Windows                                                          & Alta, ya que es familiar para administradores de Windows, pero requiere conocimientos previos de AD                           \\
    \hline
    Webmin               & Moderada, permite cierta personalización a través de temas y ajustes de interfaz, pero con limitaciones en la profundidad de las modificaciones & Moderada, la interfaz es intuitiva, pero la configuración de módulos puede ser compleja para usuarios sin experiencia técnica \\
    \hline
    samba4-manager       & Limitada, diseñada específicamente para la gestión de Samba4, con opciones de personalización limitadas                                         & Baja, debido a la complejidad de la configuración y el mantenimiento en entornos no homogéneos                                \\
    \hline
    ADWebmanager         & Limitada, diseñada para funciones comunes de AD, con mínimas opciones de personalización de interfaz                                            & Alta, diseñada para simplificar tareas comunes de gestión de AD, con una curva de aprendizaje reducida                        \\
    \hline
\end{longtable}

\subsection{Tecnologías y herramientas existentes}

En el ámbito de la gestión de AD, existen diversas tecnologías y herramientas diseñadas para facilitar esta tarea esencial en la administración de sistemas informáticos. Estas herramientas no solo permiten una gestión más eficiente de los recursos y usuarios dentro de una organización, sino que también contribuyen a mejorar la seguridad y el control de acceso.

Este epígrafe se centrará en revisar algunas de las herramientas existentes para la gestión de AD, detallando sus principales funcionalidades y características. Asimismo, se analizarán las ventajas y limitaciones de estas soluciones.


\input{sections/chapter-1/1.3.1}
\input{sections/chapter-1/1.3.2}
\input{sections/chapter-1/1.3.3}
\subsection{Tecnologías y herramientas existentes}

En el ámbito de la gestión de AD, existen diversas tecnologías y herramientas diseñadas para facilitar esta tarea esencial en la administración de sistemas informáticos. Estas herramientas no solo permiten una gestión más eficiente de los recursos y usuarios dentro de una organización, sino que también contribuyen a mejorar la seguridad y el control de acceso.

Este epígrafe se centrará en revisar algunas de las herramientas existentes para la gestión de AD, detallando sus principales funcionalidades y características. Asimismo, se analizarán las ventajas y limitaciones de estas soluciones.


\input{sections/chapter-1/1.3.1}
\input{sections/chapter-1/1.3.2}
\input{sections/chapter-1/1.3.3}
\input{sections/chapter-1/1.4}
