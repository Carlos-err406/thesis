\subsection{Protocolo Ligero de Acceso a Directorios (LDAP)}

LDAP es un estándar de protocolo utilizado para acceder y gestionar información almacenada en directorios de manera eficiente y segura. En el contexto de la administración de sistemas informáticos, LDAP desempeña un papel fundamental al facilitar la búsqueda, autenticación y gestión de usuarios, dispositivos y otros recursos \autocite{sermersheim_lightweight_2006,carter_ldap_2003}.

Este epígrafe explora los principios fundamentales de LDAP, incluyendo su arquitectura, funcionamiento y principales características. Se examinará cómo LDAP permite la estructuración jerárquica de la información mediante la utilización de entradas y atributos, lo cual facilita la organización y el acceso a datos.

\textbf{Arquitectura de LDAP}

LDAP se basa en una arquitectura cliente-servidor, donde el cliente LDAP envía solicitudes al servidor LDAP para realizar diversas operaciones, como búsquedas, actualizaciones y autenticaciones de información almacenada en el directorio. Esta arquitectura facilita la gestión centralizada y eficiente de los datos \autocite{harrison_lightweight_2006,sermersheim_lightweight_2006,carter_ldap_2003}.

\begin{itemize}
    \item \textbf{Cliente LDAP}: Es el software que realiza peticiones de búsqueda, modificación o consulta de información almacenada en el servidor LDAP.
    \item \textbf{Servidor LDAP}: Es el software que almacena la base de datos de directorio y responde a las peticiones de los clientes LDAP. El servidor LDAP gestiona y organiza la información en forma de entradas almacenadas en un árbol de directorio.
    \item \textbf{Protocolo de Comunicación}: LDAP define cómo se comunican el cliente y el servidor a través de un protocolo eficiente y ligero, diseñado principalmente para la lectura, búsqueda y modificación de información en directorios.
    \item \textbf{Eficiencia}: permite búsquedas rápidas y eficientes de información en grandes volúmenes de datos.
    \item \textbf{Seguridad}: soporta protocolos de seguridad como TLS (Transport Layer Security) para proteger la integridad y confidencialidad de los datos transmitidos.
    \item \textbf{Escalabilidad}: capacidad para manejar grandes cantidades de datos y usuarios dentro de un directorio, adaptándose a las necesidades de crecimiento de una organización.
    \item \textbf{Interoperabilidad}: estándar abierto compatible con una amplia gama de plataformas y sistemas de directorio.
\end{itemize}


\textbf{Aplicaciones de LDAP}

LDAP se utiliza ampliamente en la autenticación de usuarios, control de acceso y gestión de identidades en sistemas operativos, aplicaciones web y servicios de correo electrónico. Su flexibilidad y robustez lo convierten en una herramienta fundamental para la integración y administración de infraestructuras de TI empresariales \autocite{sermersheim_lightweight_2006,redhat_what_2022,carter_ldap_2003}.


