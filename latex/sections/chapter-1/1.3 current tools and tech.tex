\subsection{Tecnologías y herramientas existentes}

Las herramientas de gestión de AD juegan un papel crucial al permitir a los administradores de sistemas centralizar y automatizar la administración de usuarios, grupos y recursos, implementar políticas de seguridad y control de acceso, monitorear y auditar las actividades del directorio, simplificar tareas administrativas recurrentes, y garantizar la disponibilidad y consistencia de los datos. Estas capacidades son esenciales para mantener la integridad y eficiencia de los sistemas informáticos en organizaciones de cualquier tamaño.

Un ejemplo destacado es Samba4, una implementación open-source de Active Directory que, además de ofrecer estas capacidades fundamentales, proporciona compatibilidad completa con los protocolos de Microsoft AD en entornos Linux, permitiendo la integración con sistemas Windows mientras se mantiene la independencia de plataformas propietarias \autocite{imanudin_active_2019,carter_using_2007,samba_what_2019}.

Este epígrafe se centrará en revisar las tecnologías clave para la implementación de soluciones de gestión de AD, incluyendo frameworks de desarrollo web y clientes LDAP. Se analizarán las ventajas y limitaciones de estas soluciones.

% \subsubsection{Herramientas de gestión de AD}

Para gestionar efectivamente un Directorio Activo (AD), es crucial contar con herramientas que simplifiquen las tareas administrativas y ofrezcan opciones flexibles de personalización y configuración. La capacidad de personalización se refiere a la flexibilidad que ofrece la herramienta para ajustar su interfaz y funcionalidades según las necesidades específicas del usuario u organización \autocite{van_der_hoek_configurable_1999}. Por otro lado, la facilidad de uso está relacionada con la simplicidad con la que una herramienta puede ser operada y configurada, asegurando una experiencia de administración eficiente y sin complicaciones \autocite{sheppard_re-examining_2019}.

En la \autoref{table:ad-tech-comparison} se presentan algunas herramientas para la gestión de AD, resaltando sus características principales en términos de personalización y facilidad de uso \autocite{graber_stgrabersamba4-manager_2024,jerez_vicentgjad-webmanager_2024,han_remote_2024,karzynski_webmin_2014}.

\begin{longtable}{|l|p{5cm}|p{5cm}|}
    \caption{Comparación de tecnologías existentes en cuanto a capacidad de personalización y facilidad de uso
    }
    \label{table:ad-tech-comparison}                                                                                                                                                                                                                                                                       \\
    \hline
    \textbf{Herrameinta} & \textbf{Capacidad de personalización (interfaz y apariencia)}                                                                                   & \textbf{Facilidad de uso}                                                                                                     \\
    \hline
    \endfirsthead
    \hline
    RSAT                 & Limitada, la personalización se limita a ajustes mínimos dentro del entorno de Windows                                                          & Alta, ya que es familiar para administradores de Windows, pero requiere conocimientos previos de AD                           \\
    \hline
    Webmin               & Moderada, permite cierta personalización a través de temas y ajustes de interfaz, pero con limitaciones en la profundidad de las modificaciones & Moderada, la interfaz es intuitiva, pero la configuración de módulos puede ser compleja para usuarios sin experiencia técnica \\
    \hline
    samba4-manager       & Limitada, diseñada específicamente para la gestión de Samba4, con opciones de personalización limitadas                                         & Baja, debido a la complejidad de la configuración y el mantenimiento en entornos no homogéneos                                \\
    \hline
    ADWebmanager         & Limitada, diseñada para funciones comunes de AD, con mínimas opciones de personalización de interfaz                                            & Alta, diseñada para simplificar tareas comunes de gestión de AD, con una curva de aprendizaje reducida                        \\
    \hline
\end{longtable}

\subsubsection{Frameworks para el desarrollo web}

En el ámbito del desarrollo web, los frameworks proporcionan estructuras y herramientas que simplifican y aceleran la creación de aplicaciones. Estos frameworks ofrecen una base sólida para la implementación de funcionalidades complejas, facilitando la interacción entre el diseño front-end y la lógica de negocio back-end. Esta sección explora diversos frameworks destacados en el panorama actual, analizando sus características, ventajas y aplicaciones específicas en el desarrollo web moderno.

La \autoref{table:web-frameworks-comparison} presenta una comparación detallada de tres frameworks populares en el desarrollo web: SvelteKit, Next.js y Nuxt.js. Esta comparación se enfoca en varias características clave que influyen en la elección de un framework para proyectos web, tales como eficiencia, flexibilidad, escalabilidad, curva de aprendizaje y comunidad. Al analizar estos aspectos, se puede obtener una visión más clara de las fortalezas y debilidades de cada framework, ayudando a los desarrolladores a tomar decisiones informadas al seleccionar la herramienta más adecuada para sus necesidades específicas \autocite{riva_real-world_2022,lazuardy_2022_modern,kok_hands-nuxt_2020,halliday_vue_2018,wernersson_choosing_2023,kroon_celander_comparative_2024,c_ragkhitwetsagul_jscefr_2024,vepsalainen_state_2023}.

\begin{longtable}{|p{3cm}|p{3cm}|p{4cm}|p{5cm}|}
    \caption{Comparación entre frameworks de desarrollo web}
    \label{table:web-frameworks-comparison}                                                                                                                                                                                                                                                                                    \\
    \hline
    \textbf{Característica}       & \textbf{Sveltekit}                                         & \textbf{Next.js}                                                                      & \textbf{Nuxt.js}                                                                                                                      \\
    \hline
    \endfirsthead
    \textbf{Eficiencia}           & Alto rendimiento con compilación previa y sitios estáticos & Buena eficiencia con renderizado del lado del servidor y en el cliente                & Eficiencia en el desarrollo de aplicaciones web, con facilidades para la creación de aplicaciones universal y estáticamente generadas \\
    \hline
    \textbf{Flexibilidad}         & Gran flexibilidad en personalización y configuración       & Flexible para la creación de diferentes tipos de aplicaciones web                     & Flexible y adaptable a diferentes tipos de proyectos, con un enfoque en la simplicidad y facilidad de uso                             \\
    \hline
    \textbf{Escalabilidad}        & Altamente escalable para proyectos de diferentes tamaños   & Buena capacidad de escalar y manejar proyectos de gran envergadura                    & Puede escalar adecuadamente para manejar proyectos de diversos tamaños y complejidades                                                \\
    \hline
    \textbf{Curva de aprendizaje} & Muy baja, con sintaxis simple y familiar                   & Moderada, requiere familiarizarse con sus conceptos y funcionalidades                 & Moderada, especialmente para aquellos que están familiarizados con Vue.js                                                             \\
    \hline
    \textbf{Comunidad}            & En crecimiento, con soporte activo y recursos disponibles  & Amplia comunidad de desarrolladores, con gran cantidad de recursos y soporte en línea & Comunidad activa y en crecimiento, con soporte y recursos disponibles                                                                 \\
    \hline
\end{longtable}

Es importante destacar que realizar una comparación exhaustiva de todos los frameworks disponibles es complicado por varias razones:

\begin{itemize}
    \item Abundancia de opciones: Actualmente, existen cientos de frameworks de desarrollo web, cada uno con características y ventajas únicas.
    \item Complejidad inherente: Los frameworks son complejos y ofrecen una amplia gama de características, lo que dificulta una comparación exhaustiva y detallada.
    \item Variabilidad en los requisitos: Las aplicaciones web tienen requisitos diversos, lo que significa que un framework ideal para una aplicación puede no serlo para otra.
\end{itemize}

Por lo tanto, es difícil determinar que un framework sea superior a otro de manera generalizada. La elección del mejor framework para una aplicación específica depende en gran medida de los requisitos particulares de cada aplicación y de las preferencias del equipo de desarrollo. En última instancia, la selección de un framework adecuado se basa en su capacidad para resolver los problemas específicos de desarrollo que se presentan en el contexto de la aplicación deseada.

\subsubsection{Clientes LDAP}

LDAP se ha consolidado como un estándar esencial en la administración de AD. Su capacidad para gestionar y acceder a información jerárquica de manera eficiente ha hecho que múltiples aplicaciones y servicios adopten clientes LDAP para interactuar con los directorios.

En este epígrafe, se proporciona una definición de cliente LDAP basada en la literatura, se exponen sus características principales, se explica la importancia del uso de estos clientes y se exploran diversos clientes LDAP disponibles, analizando sus características, ventajas y limitaciones.

\textbf{¿Qué es un cliente LDAP?}

Un cliente LDAP es una aplicación o herramienta que permite a los usuarios y sistemas interactuar con un servidor de AD. Su función principal es facilitar la comunicación con el directorio, permitiendo que se realicen operaciones como búsquedas, modificaciones, adiciones y eliminaciones de entradas en la base de datos del directorio. Estos clientes actúan como intermediarios que traducen las solicitudes de los usuarios o aplicaciones a un formato comprensible para el directorio, simplificando la interacción y mejorando la eficiencia en la administración de datos.

\textbf{Capacidades de los clientes LDAP}
\begin{itemize}
    \item \textbf{Búsquedas}: Los clientes LDAP pueden realizar búsquedas en el directorio para encontrar información específica basada en varios criterios. Esto es crucial para aplicaciones que necesitan recuperar datos de manera rápida y eficiente.
    \item \textbf{Modificaciones}: Permiten actualizar la información existente en el directorio. Las modificaciones pueden incluir cambios en atributos de una entrada o actualizaciones de múltiples entradas simultáneamente.
    \item \textbf{Adiciones}: Los clientes LDAP facilitan la adición de nuevas entradas en el directorio. Esto es útil para la incorporación de nuevos usuarios, dispositivos o cualquier otra entidad que necesite ser gestionada dentro del directorio.
    \item \textbf{Eliminaciones}: También soportan la eliminación de entradas del directorio, ayudando a mantener la información actualizada y eliminando datos obsoletos o incorrectos.
\end{itemize}

\textbf{Abstracción de la lógica}

La abstracción de la lógica en la interacción con el directorio mediante un cliente LDAP es fundamental por varias razones:

\begin{itemize}
    \item \textbf{Simplicidad y Eficiencia}: Al utilizar un cliente LDAP, los desarrolladores y administradores no necesitan conocer los detalles específicos del protocolo LDAP. Esto simplifica el desarrollo y la administración, permitiendo centrarse en la lógica de negocio en lugar de en los detalles técnicos.
    \item \textbf{Interoperabilidad}: Los clientes LDAP son compatibles con múltiples sistemas y aplicaciones, lo que facilita la integración de diversas soluciones en una infraestructura común. Esto es crucial para la interoperabilidad entre sistemas heterogéneos.
    \item \textbf{Seguridad}: Al centralizar las peticiones a través de un cliente LDAP, es posible implementar políticas de seguridad consistentes, como autenticación y autorización, garantizando que solo usuarios y aplicaciones autorizadas puedan acceder y modificar la información del directorio.
\end{itemize}

\textbf{Comparación entre clientes LDAP existentes}

La elección del cliente LDAP adecuado es crucial para gestionar eficientemente un AD. Con la creciente variedad de opciones disponibles, es fundamental entender las diferencias y capacidades de cada cliente LDAP.
En la \autoref{table:ldap-client-comparison} se comparan varios clientes LDAP, evaluando aspectos clave que pueden influir en la elección de una solución:

\begin{itemize}
    \item \textbf{Dependencia de ldapjs}: Indica si el cliente LDAP tiene dependencia de la biblioteca ldapjs. Esto es relevante debido a la descontinuación de ldapjs, que hasta el 14 de Mayo de 2024 era ampliamente utilizada. Esta evaluación asegura la selección de soluciones que ofrecen una base estable y sostenible, minimizando riesgos asociados con la obsolescencia y garantizando la compatibilidad a largo plazo con el ecosistema LDAP.
    \item \textbf{Soporte TypeScript}: La compatibilidad con TypeScript no solo permite el desarrollo más seguro y estructurado de aplicaciones modernas, sino que también mejora significativamente la detección de errores en tiempo de desarrollo.
    \item \textbf{Calidad de la documentación}: La disponibilidad y calidad de la documentación afecta directamente la rapidez con la que los desarrolladores pueden familiarizarse con el cliente LDAP y resolver problemas.
    \item \textbf{Comunidad}: Una comunidad activa y un soporte sólido aseguran que los problemas se resuelvan rápidamente y que el cliente LDAP se mantenga actualizado con las mejores prácticas. El nivel de actividad de la comunidad y la frecuencia de las actualizaciones pueden variar, impactando la estabilidad y la confianza en el uso a largo plazo del cliente.
    \item \textbf{Facilidad de uso}: Evalúa la simplicidad y la curva de aprendizaje del cliente LDAP. Una alta facilidad de uso reduce el tiempo de integración y minimiza errores durante la implementación. Esto incluye si el cliente provee abstracciones y/o funciones de alto nivel, lo cual puede facilitar significativamente su uso.
\end{itemize}

\newgeometry{vmargin=1.4cm,hmargin=0.8cm}
\begin{landscape}
    \begin{longtable}{|l|p{3cm}|p{2.5cm}|p{6cm}|p{5cm}|p{6cm}|}
        \caption{Comparación entre distintos clientes LDAP (Marzo 2025)}
        \label{table:ldap-client-comparison}                                                                                                                                                                                                                                                                                                                                                                                                                \\
        \hline
        \textbf{Cliente LDAP} & \textbf{Dependencia de ldapjs} & \textbf{Soporta TypeScript} & \textbf{Calidad de la documentación}                                                                                                       & \textbf{Comunidad y soporte (en julio 2024)}                                           & \textbf{Facilidad de uso}                                                                                              \\
        \hline
        \endfirsthead
        ldapts                & No                             & Sí                          & Excelente, documentación completa y fácil de entender, con ejemplos detallados.                                                            & Activa y sólida, última publicación en Enero 2025 y más de 90 mil descargas semanales. & Alta, fácil de usar y aprender, con una curva de aprendizaje baja. Provee abstracciones para la composición de filtros \\
        \hline
        ldap-client           & No                             & No                          & Buena, documentación adecuada, pero básica. Cubre la mayoría de los casos de uso comunes, aunque carece de ejemplos avanzados.             & Baja, última publicación 2016, con 20 descargas semanales.                             & Media, requiere algún tiempo de aprendizaje, pero es manejable. No provee abstracciones para la composición de filtros \\
        \hline
        activedirectory       & Sí                             & No                          & Buena, documentación adecuada, con suficiente información para la configuración y uso básico, aunque podría mejorar en detalle y ejemplos. & Moderada, última publicación 2016, con 15 mil descargas semanales                      & Media, interfaz familiar, provee funciones de más alto nivel  específicas  para la busqueda de usuarios y grupos.      \\
        \hline
        ldap-ts-client        & Sí                             & Sí                          & Escasa, documentación con poca información disponible y pocos ejemplos.                                                                    & Baja, última publicación 2022, con 51 descargas semanales                              & Baja, debido a la falta de documentacion.                                                                              \\
        \hline
    \end{longtable}
\end{landscape}
\restoregeometry
