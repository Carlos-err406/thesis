\begin{otherlanguage}{spanish}

    \begin{abstract}
        Esta tesis presenta una aplicación web de código abierto para la gestión y autenticación de usuarios basada en Directorio Activo a través de LDAP. La solución aborda las limitaciones de sus predecesores al ofrecer una consola de administración centralizada altamente configurable y adaptable a las necesidades organizacionales. El desarrollo siguió un enfoque sistemático que incluyó análisis de requisitos, selección de tecnologías e implementación de funcionalidades clave.

        Aspectos principales incluyen una integración robusta del cliente LDAP, una interfaz de usuario responsiva y flujos de gestión de usuarios. La aplicación demuestra ventajas significativas en términos de flexibilidad y facilidad de uso en comparación con sus predecesores. Los resultados de las pruebas confirman la viabilidad del enfoque, con la implementación exitosa de características críticas.

        Este trabajo contribuye al campo de la gestión de usuarios y directorios al establecer un marco para futuras investigaciones y desarrollos. La solución propuesta se posiciona como una alternativa valiosa para organizaciones que buscan mejorar su gestión de identidades en un panorama digital en constante evolución.
        \vfill
        \textbf{Palabras clave}: gestión de usuarios, código abierto, Directorio Activo, LDAP, autenticación, personalización, aplicación web.
    \end{abstract}

\end{otherlanguage}
