\begin{otherlanguage}{spanish}

    \begin{abstract}
        Esta tesis desarrolla una aplicación web de código abierto para la gestión de Directorio Activo a través de LDAP con Samba4 como controlador de dominio, abordando las limitaciones de soluciones existentes como ADWebmanager. La investigación establece una base teórica para la gestión centralizada de usuarios y servicios de directorio, analizando herramientas y tecnologías actuales. La solución propuesta implementa un modelo de dominio completo con arquitectura en capas, utilizando SvelteKit y ldapts para la integración con AD.

        La aplicación proporciona funcionalidad robusta del cliente LDAP, interfaz de usuario responsiva y flujos completos de gestión de usuarios. Las características clave incluyen despliegue automatizado mediante CI/CD y contenedores de Docker, junto con documentación técnica generada automáticamente. Las metodologías de prueba incluyeron pruebas unitarias y de integración con Vitest, y pruebas E2E con Playwright, validando todas las operaciones críticas.

        Los resultados demuestran la implementación exitosa de funcionalidades principales, incluyendo creación, modificación y eliminación de recursos. La solución ofrece mejoras significativas en flexibilidad, seguridad y facilidad de uso en comparación con herramientas existentes. Este trabajo contribuye al campo de la gestión de usuarios y directorios al establecer un marco para futuras investigaciones y desarrollos.

        La tesis concluye que la solución propuesta representa una alternativa valiosa para organizaciones que buscan mejorar sus sistemas de gestión de identidades, particularmente en entornos que requieren altos niveles de personalización y seguridad.
        \vfill
        \textbf{Palabras clave}: gestión de usuarios, código abierto, Directorio Activo, LDAP, SvelteKit, despliegue automatizado, pruebas.
    \end{abstract}

\end{otherlanguage}
