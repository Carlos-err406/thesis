\section{Introducción}

La gestión de usuarios y la autenticación en entornos de red son procesos esenciales en cualquier organización \autocite{thakur_user_2015-1, josang_local_2015, kizza_access_2024}. Los directorios activos desempeñan un papel clave al centralizar y asegurar el control de accesos a los diferentes servicios y recursos empresariales \autocite{kizza_access_2024}. Actualmente, estas herramientas son ampliamente utilizadas en empresas de diversos tamaños para gestionar usuarios, grupos y unidades organizativas de manera unificada (\autoref{fig:companies-using-ad}).

Un Directorio Activo (AD por sus siglas en inglés) es una base de datos jerárquica utilizada para almacenar y gestionar información sobre los recursos de la red, como usuarios, dispositivos y servicios. Proporciona una estructura centralizada que permite a los administradores controlar permisos y acceso a los recursos de manera segura y eficiente  \autocite{harrison_lightweight_2006,thakur_user_2015,carter_ldap_2003}. 

En este contexto, el protocolo LDAP (Lightweight Directory Access Protocol) se utiliza para interactuar con los directorios activos, facilitando la búsqueda, consulta y modificación de la información almacenada en ellos \autocite{harrison_lightweight_2006,sermersheim_lightweight_2006,bartlett_samba_2005,voglmaier_abcs_2003,redhat_what_2022,janice_ldap_2023}.

Existen dos grandes grupos de directorios activos en el mercado: los de pago y los de código abierto. Las soluciones comerciales, como las ofrecidas por Microsoft, destacan por su facilidad de uso y alto nivel de integración, pero también generan una dependencia tecnológica significativa y pueden no ser viables para organizaciones con presupuestos limitados. En contraste, los directorios activos de código abierto que eliminan la necesidad de costosas licencias y ofrecen independencia tecnológica \autocite{thakur_user_2015-1,bartlett_samba_2005,imanudin_active_2019}.

Entre las soluciones libres, Samba4 emerge como una solución open-source que permite implementar un controlador de dominio de AD en infraestructura Linux, ofreciendo compatibilidad con los protocolos de Microsoft AD. 

Un ejemplo práctico de adopción de Samba4 se encuentra en la empresa Avangenio. Sin embargo, las herramientas de administración incluidas con Samba4 están limitadas a la línea de comandos, lo que las hace menos accesibles en contextos de administración de redes empresariales \autocite{bartlett_samba_2005,imanudin_active_2019,samba_what_2019}.

Para superar esta limitación, se desarrollaron soluciones web como ADwebmanager \autocite{jerez_vicentgjad-webmanager_2024}, un fork del proyecto Samba4-manager \autocite{graber_stgrabersamba4-manager_2024} que actualiza el código fuente a Python 3. Aunque ADwebmanager representó un avance en la accesibilidad comparado con la gestión a través de la línea de comandos, presentó varias limitaciones que afectaron su escalabilidad y usabilidad:

\begin{itemize}
    \item Interfaz no responsiva, dificultando su uso en dispositivos móviles.
    \item Requiere recargas completas de la página para actualizar la interfaz.
    \item Frontend basado en templates con elementos dinámicos difíciles de modificar.
    \item Vulnerabilidad a ataques de fuerza bruta en la página de autenticación.
    \item Inconsistencias en la interfaz de usuario que afectan la experiencia de usuario.
\end{itemize}

Estas limitaciones llevan a identificar como problema a resolver: las deficiencias de ADWebmanager en términos de seguridad y experiencia de usuario dificultan la gestión eficiente de los recursos empresariales.

Para solucionar este problema se tiene como objeto de estudio los las herramientas de administración de AD. El campo de acción se delimita a los sistemas web de SLCA para la gestión de AD, con especial atención a las soluciones web para Samba4.

Como hipótesis se plantea que desarrollar una herramienta web de código abierto para la gestión de AD basada en Samba4 que ofrezca un mayor nivel de personalización a través de archivos de configuración permitirá mejorar la adaptabilidad y facilidad de uso, sin comprometer la seguridad y estabilidad, en comparación con ADWebmanager.

Para demostrar esta hipótesis se plantea como objetivo general crear una consola de administración de código abierto para AD que demuestre una mejora en la personalización y facilidad de uso en comparación con las herramientas de gestión existentes.

A partir de este objetivo general, se derivan los siguientes objetivos específicos y tareas:

\begin{enumerate}[label=\arabic*., itemindent=*, leftmargin=*]

    \item Analizar los requisitos de la aplicación y la personalización del sistema:
          \begin{enumerate}[label=\arabic{enumi}.\arabic*., leftmargin=*]
              \item Documentar los requisitos funcionales y no funcionales que debe cumplir la aplicación.
              \item Analizar diferentes casos de uso para identificar las opciones de personalización.
              \item Documentar los requisitos de personalización, incluyendo la interfaz de usuario y ajustes de seguridad.
          \end{enumerate}

    \item Seleccionar tecnologías adecuadas:
          \begin{enumerate}[label=\arabic{enumi}.\arabic*., leftmargin=*]
              \item Evaluar diferentes clientes LDAP disponibles, considerando factores como compatibilidad, rendimiento y facilidad de integración.
              \item Elegir el cliente LDAP que mejor se alinee con los requisitos funcionales y no funcionales previamente definidos.
          \end{enumerate}

    \item Implementar la arquitectura y funciones básicas de la aplicación:
          \begin{enumerate}[label=\arabic{enumi}.\arabic*., leftmargin=*]
              \item Establecer la arquitectura base del proyecto y configurar el ambiente de desarrollo necesario.
              \item Configurar el cliente LDAP seleccionado y las herramientas asociadas para iniciar el desarrollo.
              \item Desarrollar mecanismos de autenticación que interactúen con el AD utilizando el cliente LDAP seleccionado.
              \item Implementar funcionalidades críticas para la gestión de usuarios y grupos utilizando el cliente LDAP, incluyendo operaciones de lectura, eliminación y actualización.
          \end{enumerate}

    \item Realizar pruebas para asegurar el correcto funcionamiento del sistema:
          \begin{enumerate}[label=\arabic{enumi}.\arabic*., leftmargin=*]
              \item Diseñar y ejecutar pruebas de integración que verifiquen el correcto funcionamiento del sistema en su conjunto, desde la autenticación hasta la gestión de recursos.
              \item Documentar los resultados de las pruebas y realizar los ajustes necesarios basados en los hallazgos.
          \end{enumerate}

    \item Generar documentación técnica:
          \begin{enumerate}[label=\arabic{enumi}.\arabic*., leftmargin=*]
              \item Automatizar generación de documentación técnica utilizando los esquemas JSON existentes para describir la estructura y validación de los archivos de configuración.
              \item Integrar el proceso de generación de documentación en un pipeline de CI/CD, asegurando la actualización automática con cada cambio en los esquemas.
              \item Publicar la documentación como un sitio web estático accesible públicamente, proporcionando una referencia técnica completa y actualizada.
          \end{enumerate}

    \item Desplegar aplicación:
          \begin{enumerate}[label=\arabic{enumi}.\arabic*., leftmargin=*]
              \item Configurar entornos de desarrollo, pruebas y producción utilizando Docker, garantizando consistencia entre diferentes entornos.
              \item Implementar un pipeline de CI/CD que automatice la construcción, pruebas y despliegue de la aplicación.
          \end{enumerate}
\end{enumerate}


Como valor práctico, este trabajo aporta una solución de gestión de AD que mejora significativamente la seguridad y usabilidad respecto a herramientas existentes. La implementación de mecanismos robustos de autenticación, incluyendo protección contra ataques de fuerza bruta y validación de credenciales segura, garantiza un sistema confiable para las organizaciones. Adicionalmente, la interfaz web responsiva e intuitiva facilita las tareas de administración, permitiendo gestionar usuarios, grupos y unidades organizativas de manera centralizada. La herramienta también ofrece personalización mediante archivos JSON/YAML para adaptarse a diferentes políticas organizacionales, y reduce costos al eliminar la dependencia de soluciones comerciales propietarias.

Estas características la convierten en una alternativa flexible y accesible, especialmente útil para organizaciones con recursos limitados o necesidades específicas.



La estructura del informe se organiza de la siguiente manera:

Capítulo 1 Fundamentación teórica. En este capítulo se presenta una explicación teórica sobre los conceptos y tecnologías fundamentales para la gestión de usuarios y AD. Se analizarán en profundidad temas como la importancia de la gestión de usuarios en entornos digitales, los principios de seguridad y autenticación, y el funcionamiento de los directorios activos y LDAP. También se abordarán las diferentes herramientas existentes para la gestión de AD, sus ventajas y limitaciones, y se establecerá el marco teórico que sustenta la propuesta de solución planteada en este trabajo.

Capítulo 2 Descripción de la propuesta de solución. Este capítulo aporta una explicación detallada sobre la propuesta de solución a los problemas identificados en la gestión de AD. Se describirá la arquitectura de la aplicación web de código abierto propuesta, sus funcionalidades principales, y cómo se abordarán los requisitos de personalización y simplicidad de despliegue. Además, se explicarán las decisiones de diseño y las tecnologías seleccionadas (SvelteKit y ldapts) para el desarrollo de la herramienta, así como los procesos de documentación y despliegue automatizado.

Capítulo 3 Validación de la propuesta de solución. Este capítulo está dedicado a la validación de la propuesta de solución mediante la realización de pruebas exhaustivas. Se diseñarán y ejecutarán pruebas de integración para verificar el correcto funcionamiento del sistema en su conjunto, desde la autenticación hasta la gestión de recursos. Los resultados de estas pruebas se documentarán y se realizarán los ajustes necesarios basados en los hallazgos.

