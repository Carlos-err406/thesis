A partir de este objetivo general, se derivan los siguientes objetivos específicos y tareas:

\begin{enumerate}[label=\arabic*., itemindent=*, leftmargin=*]

    \item Analizar los requisitos de la aplicación y la personalización del sistema:
          \begin{enumerate}[label=\arabic{enumi}.\arabic*., leftmargin=*]
              \item Documentar los requisitos funcionales y no funcionales que debe cumplir la aplicación.
              \item Analizar diferentes casos de uso para identificar las opciones de personalización.
              \item Documentar los requisitos de personalización, incluyendo la interfaz de usuario y ajustes de seguridad.
          \end{enumerate}

    \item Seleccionar tecnologías adecuadas:
          \begin{enumerate}[label=\arabic{enumi}.\arabic*., leftmargin=*]
              \item Evaluar diferentes clientes LDAP disponibles, considerando factores como compatibilidad, rendimiento y facilidad de integración.
              \item Elegir el cliente LDAP que mejor se alinee con los requisitos funcionales y no funcionales previamente definidos.
          \end{enumerate}

    \item Implementar la arquitectura y funciones básicas de la aplicación:
          \begin{enumerate}[label=\arabic{enumi}.\arabic*., leftmargin=*]
              \item Establecer la arquitectura base del proyecto y configurar el ambiente de desarrollo necesario.
              \item Configurar el cliente LDAP seleccionado y las herramientas asociadas para iniciar el desarrollo.
              \item Desarrollar mecanismos de autenticación que interactúen con el AD utilizando el cliente LDAP seleccionado.
              \item Implementar funcionalidades críticas para la gestión de usuarios y grupos utilizando el cliente LDAP, incluyendo operaciones de lectura, eliminación y actualización.
          \end{enumerate}

    \item Realizar pruebas para asegurar el correcto funcionamiento del sistema:
          \begin{enumerate}[label=\arabic{enumi}.\arabic*., leftmargin=*]
              \item Diseñar y ejecutar pruebas de integración que verifiquen el correcto funcionamiento del sistema en su conjunto, desde la autenticación hasta la gestión de recursos.
              \item Documentar los resultados de las pruebas y realizar los ajustes necesarios basados en los hallazgos.
          \end{enumerate}

    \item Generar documentación técnica:
          \begin{enumerate}[label=\arabic{enumi}.\arabic*., leftmargin=*]
              \item Automatizar generación de documentación técnica utilizando los esquemas JSON existentes para describir la estructura y validación de los archivos de configuración.
              \item Integrar el proceso de generación de documentación en un pipeline de CI/CD, asegurando la actualización automática con cada cambio en los esquemas.
              \item Publicar la documentación como un sitio web estático accesible públicamente, proporcionando una referencia técnica completa y actualizada.
          \end{enumerate}

    \item Desplegar aplicación:
          \begin{enumerate}[label=\arabic{enumi}.\arabic*., leftmargin=*]
              \item Configurar entornos de desarrollo, pruebas y producción utilizando Docker, garantizando consistencia entre diferentes entornos.
              \item Implementar un pipeline de CI/CD que automatice la construcción, pruebas y despliegue de la aplicación.
          \end{enumerate}
\end{enumerate}
