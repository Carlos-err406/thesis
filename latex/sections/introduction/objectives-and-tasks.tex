A partir de este objetivo general, se derivan los siguientes objetivos específicos y tareas:

\begin{enumerate}[label=\arabic*., itemindent=*, leftmargin=*]

    \item Analizar los requisitos de la aplicación y la personalización del sistema:
          \begin{enumerate}[label=\arabic{enumi}.\arabic*., leftmargin=*]
              \item Documentar los requisitos funcionales y no funcionales que debe cumplir la aplicación.
              \item Analizar diferentes casos de uso para identificar las opciones de personalización.
              \item Documentar los requisitos de personalización, incluyendo la interfaz de usuario y ajustes de seguridad.
          \end{enumerate}

    \item Seleccionar tecnologías adecuadas:
          \begin{enumerate}[label=\arabic{enumi}.\arabic*., leftmargin=*]
              \item Evaluar diferentes clientes LDAP disponibles en el mercado, considerando factores como compatibilidad, rendimiento y facilidad de integración.
              \item Elegir el cliente LDAP que mejor se alinee con los requisitos funcionales y no funcionales previamente definidos.
          \end{enumerate}

    \item Seleccionar tecnologías adecuadas:
          \begin{enumerate}[label=\arabic{enumi}.\arabic*., leftmargin=*]
              \item Evaluar diferentes clientes LDAP disponibles en el mercado, considerando factores como compatibilidad, rendimiento y facilidad de integración.
              \item Elegir el cliente LDAP que mejor se alinee con los requisitos funcionales y no funcionales previamente definidos.
          \end{enumerate}

    \item Implementar la arquitectura y funciones básicas de la aplicación:
          \begin{enumerate}[label=\arabic{enumi}.\arabic*., leftmargin=*]
              \item Establecer la arquitectura base del proyecto y configurar el ambiente de desarrollo necesario.
              \item Configurar el cliente LDAP seleccionado y las herramientas asociadas para iniciar el desarrollo.
              \item Desarrollar mecanismos de autenticación que interactúen con el AD utilizando el cliente LDAP seleccionado.
              \item Implementar funcionalidades críticas para la gestión de usuarios y grupos utilizando el cliente LDAP, incluyendo operaciones de lectura, eliminación y actualización.
          \end{enumerate}

    \item Realizar pruebas para asegurar el correcto funcionamiento del sistema:
          \begin{enumerate}[label=\arabic{enumi}.\arabic*., leftmargin=*]
              \item Diseñar y ejecutar pruebas de integración que verifiquen el correcto funcionamiento del sistema en su conjunto, desde la autenticación hasta la gestión de recursos.
              \item Documentar los resultados de las pruebas y realizar los ajustes necesarios basados en los hallazgos.
              \item Extender el conjunto de pruebas de integración para abarcar nuevas funcionalidades y garantizar la estabilidad y compatibilidad del sistema ante cambios y actualizaciones futuras.
          \end{enumerate}

    \item Simplificar y documentar el proceso de despliegue:
          \begin{enumerate}[label=\arabic{enumi}.\arabic*., leftmargin=*]
              \item Identificar estrategias y herramientas que simplifiquen el proceso de instalación y configuración inicial de la aplicación.
              \item Utilizar contenedores Docker para simplificar el proceso de puesta en marcha de la aplicación.
              \item Elaborar documentación detallada del proceso de despliegue.
          \end{enumerate}

    \item Desplegar documentación:
          \begin{enumerate}[label=\arabic{enumi}.\arabic*., leftmargin=*]
              \item Crear y estructurar la documentación técnica que incluya la descripción del sistema, la arquitectura, y las guías de desarrollo.
              \item Documentar referencias de API y/o archivos de configuracion que sean claras y accesibles.
              \item Publicar la documentación en un sitio web accesible.
          \end{enumerate}

    \item Desplegar demo:
          \begin{enumerate}[label=\arabic{enumi}.\arabic*., leftmargin=*]
              \item Configurar el servidor donde se va a desplegar.
              \item Configurar Docker para simplificar el despliegue.
              \item Configurar CI/CD para despliegues automáticos.
              \item Ejecutar el primer despliegue exitoso.
          \end{enumerate}
\end{enumerate}
