
La estructura del informe se organiza de la siguiente manera:

Capítulo 1 Fundamentación teórica. En este capítulo se presenta una explicación teórica sobre los conceptos y tecnologías fundamentales para la gestión de usuarios y AD. Se analizarán en profundidad temas como la importancia de la gestión de usuarios en entornos digitales, los principios de seguridad y autenticación, y el funcionamiento de los directorios activos y LDAP. También se abordarán las diferentes herramientas existentes para la gestión de AD, sus ventajas y limitaciones, y se establecerá el marco teórico que sustenta la propuesta de solución planteada en este trabajo.

Capítulo 2 Descripción de la propuesta de solución. Este capítulo aporta una explicación detallada sobre la propuesta de solución a los problemas identificados en la gestión de AD. Se describirá la arquitectura de la aplicación web de código abierto propuesta, sus funcionalidades principales, y cómo se abordarán los requisitos de personalización y simplicidad de despliegue. Además, se explicarán las decisiones de diseño y las tecnologías seleccionadas para el desarrollo de la herramienta, así como los beneficios esperados en términos de seguridad, estabilidad y facilidad de uso.

Capítulo 3 Validación de la propuesta de solución. Este capítulo está dedicado a la validación de la propuesta de solución mediante la realización de pruebas exhaustivas. Se diseñarán y ejecutarán pruebas de integración para verificar el correcto funcionamiento del sistema en su conjunto, desde la autenticación hasta la gestión de recursos. Los resultados de estas pruebas se documentarán y se realizarán los ajustes necesarios basados en los hallazgos.
