\subsubsection{Stack tecnológico}

El stack tecnológico elegido para el desarrollo de la aplicación fue fundamental para garantizar tanto el rendimiento como la mantenibilidad del sistema. A continuación, se describen los componentes principales del stack:


\textbf{Framework de desarrollo: SvelteKit}

SvelteKit se ha seleccionado como el framework principal debido a su capacidad para crear aplicaciones rápidas con una experiencia de desarrollo eficiente. SvelteKit ofrece ventajas como el renderizado en el lado del servidor (SSR), la generación de sitios estáticos (SSG), y una integración sencilla con herramientas modernas como Vite. Además, su enfoque en la eliminación del tiempo de ejecución permite que las aplicaciones sean ligeras y rápidas, lo cual es esencial para la experiencia del usuario.

\textbf{Cliente LDAP: ldapts}

Para la comunicación con el servidor LDAP, se ha optado por ldapts, un cliente LDAP para Node.js que ofrece una API asincrónica basada en promesas y con soporte para typescript. Esta herramienta es ideal para interactuar con AD, permitiendo operaciones como búsqueda, adición, modificación y eliminación de entradas en el directorio. Su uso simplifica la integración de LDAP en la aplicación, garantizando la seguridad y eficiencia necesarias para la gestión de usuarios.
