\subsubsection{Problemas frecuentes y soluciones en la arquitectura de la aplicación}

Este epígrafe aborda los problemas comunes que pueden surgir en el desarrollo de aplicaciones web y presenta soluciones basadas en la arquitectura adoptada. Se exploran aspectos clave como la seguridad, el rendimiento, la escalabilidad y la integración de servicios externos, destacando cómo la elección adecuada de tecnologías puede ayudar a mitigar estos desafíos.

\textbf{Integración de APIs y servicios externos}: las aplicaciones web a menudo requieren la integración con servicios externos. En este caso, la única integración externa es con AD, que se realiza a través del cliente LDAP proporcionado por la librería ldapts. Esta librería permite una conexión eficiente con el directorio, aprovechando su naturaleza asíncrona para mejorar el rendimiento y los tiempos de carga.

\textbf{Rendimiento y tiempos de carga}: problemas de rendimiento y tiempos de carga pueden afectar la experiencia del usuario. SvelteKit, con su enfoque de generación de sitios estáticos y pre-renderización, ayuda a mejorar la velocidad de carga. La integración con ldapts, una librería asíncrona para la comunicación con AD, mejora el rendimiento al manejar operaciones de directorio de manera eficiente sin bloquear el hilo principal. Esto asegura que las consultas LDAP no impacten negativamente en el rendimiento global de la aplicación.

\textbf{Escalabilidad y mantenibilidad}: con el crecimiento de la aplicación, la escalabilidad y mantenibilidad son esenciales. La arquitectura en capas implementada con SvelteKit facilita una separación clara de responsabilidades, mejorando la organización y la evolución del código. La capa de presentación, la capa de negocio y la capa de datos están claramente definidas, permitiendo un desarrollo modular y la fácil adaptación a nuevas funcionalidades. Esta estructura en capas asegura que la aplicación pueda escalar de manera efectiva y mantenerse fácilmente a medida que se amplían los requisitos.

\textbf{Seguridad de la información y acceso}: La gestión de identidades y el control de acceso son esenciales para asegurar la aplicación web. La integración con AD a través del cliente LDAP de ldapts permite una autenticación centralizada y segura para los usuarios. Para reforzar la seguridad, se utiliza LDAPS (LDAP sobre SSL/TLS), que cifra la comunicación entre la aplicación y el servidor LDAP, protegiendo los datos de autenticación y los permisos de acceso contra interceptaciones y ataques. Este enfoque garantiza un manejo riguroso de credenciales y permisos, manteniendo la integridad y confidencialidad de la información.