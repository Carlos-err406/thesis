\section{Conclusiones}

Al finalizar trabajo se llegó a las siguientes conclusiones:

A pesar de la robustez de las soluciones existentes, muchas de ellas presentan limitaciones en cuanto a personalización y facilidad de uso. La propuesta de una consola de administración que permita a los administradores gestionar usuarios y grupos de manera centralizada, al tiempo que se adapta a las necesidades específicas de cada organización, representa un avance significativo en este campo.

Los objetivos planteados se han cumplido mediante un enfoque sistemático que incluyó la identificación de requisitos, la selección de tecnologías adecuadas, la implementación de funcionalidades críticas y la realización de pruebas. Los resultados obtenidos demuestran que la solución propuesta no solo es viable, sino que también ofrece ventajas en términos de adaptabilidad.

En conclusión, esta investigación no solo contribuye al campo de la gestión de usuarios y directorios, sino que también establece un marco para futuras investigaciones y desarrollos en este ámbito. La combinación de flexibilidad, y facilidad de uso posiciona a la solución propuesta como una alternativa para organizaciones que buscan mejorar su gestión de identidades en un entorno digital en constante evolución.
